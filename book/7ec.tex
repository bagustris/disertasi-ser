\chapter{Comparative Analysis}
This chapter aims at summarizing the results obtained in this research and
comparing with others.


\section{Comparison within this study}
To this end, several methods have been proposed, and several results have been
obtained. The summary of the results could be used to evaluate the research and
its trend. Comparing one method to another, either from the feature side or
classifier/model side, could be used to judge the effectiveness of the method.
The use of a single metric CCC permits to track the performance of
different methods precisely.

Improvements in CCC scores were obtained using different methods described from
Chapter 4 to Chapter 6. Table \ref{tab:sum_ch7} shows the averaged CCC scores
among valence, arousal, and dominance using different acoustic features,
linguistic features, and classifiers on the IEMOCAP dataset. Clearly, it shows
the gradation of the performance scores (average CCC [$CCC_{ave}$]) from
unimodal acoustic SER to bimodal acoustic-linguistic SER.

\begin{table}[htbp]
    \begin{center}
\caption{Reported results on the IEMOCAP dataset test set (Session 5); the
number inside bracket represents the number of layers; sil: silence}
    \begin{tabular}{c c c c}
    \hline   
Acoustic & Linguistic & Classifier & $CCC_{avg}$ \\
\hline \hline   
GeMAPS LLD & - & LSTM (3) &  0.382 \\
pAA LLD    & - & LSTM (3) &  0.354 \\
pAA\_D LLD  & - & LSTM (3) &  0.370 \\
GeMAPS HSF & - & LSTM (3) &  0.389 \\
pAA HSF    & - & LSTM (3) &  0.384 \\
pAA\_D HSF  & - & LSTM (3) &  0.413 \\
pAA\_D HSF  & - & MLP (6) & 0.452 \\
pAA\_D HSF sil-removed & -  & MLP (6) & 0.459 \\
pAA\_D HSF + sil & - & MLP (6) & 0.466 \\
- & WE &  LSTM (3) &    0.361 \\
- & word2vec    &   LSTM (3)   &   0.386  \\
- & GloVe &   LSTM  (3)  &   0.392 \\
- & FastText &    LSTM  (3)  &   0.384 \\
- & BERT    &  LSTM (3)    &   0.375 \\
pAA HSF & GloVe &   LSTM (3) -- LSTM (3) [early fusion]    &   0.508 \\
pAA\ HSF + sil (HSF2) & Glove & LSTM (3) -- SVM [late fusion]      
&   0.532 \\
    \hline
    \end{tabular}
    \label{tab:sum_ch7}
    \end{center}
\end{table}

The first eight rows in Table \ref{tab:sum_ch7} shows dimensional SER from
acoustic features only. HSF consistently obtained higher CCC scores than LLD.
An optimization of this feature set can be achieved using MLP architecture with
a deeper layer (a maximum score was obtained using five layers). A comparison
of ignoring (keeping) silence, removing silence, and utilizing silence as an
additional feature showed that the latter two methods are better than the
first. Since linguistic-only dimensional SER is not the focus of this study,
the discussion of linguistic-only dimensional SER is not discussed thoroughly.


Instead, the use of linguistic information, in addition to acoustic
information, improved the performance scores. Using pAA feature set with LSTM
as the baseline, the early fusion method improved the average CCC score from
0.400 to 0.508.  Furthermore, the late fusion method improved the early fusion
method from 0.508 to 0.532. As suggested in the next chapter, there is a
bottleneck between acoustic and linguistic processing in bimodal SER fusion;
the process needs to wait for ASR outputs in real application for information
fusion. However, bimodal fusion consistently gains higher scores than
acoustic-only SER. Although there is a room for improvement, this study limits
the discussion to this point since the goal is studying the fuse of acoustic
and linguistic information for dimensional emotion recognition.  Further
investigations are recommended in the next chapter's Future research direction
section. 


\section{Comparison with other studies}
One of the motivations to use the dimensional model over the categorical model
is that only a little research has been conducted using this emotion model.
This small number of research leads to the difficulties of comparing this
research to similar studies. In INTERSPEECH 2020, 11 papers proposed bimodal
acoustic-linguistic fusion for SER. Among these papers, only four papers
evaluated dimensional SER.  However, no paper reported the results in CCC
score.  

Table \ref{tab:compare} compares this research with others. Although the exact
condition cannot be performed for ideal comparison, the performance of valence
(V), arousal (A), and dominance (D) in CCC scores can be used to judge the
rough performance among several methods. The first four rows are the results
obtained in this study. The rest are from other research with different
datasets and methods.

The closest comparison can be made between this research and the ones proposed
by Zhao et al. \cite{Zhao2018a,Zhang2019}. In both papers, the authors proposed
to fuse acoustic features with gender and age information. Using the other
non-linguistic information, they improved the CCC scores of dimensional SER
except for valence prediction. The fusion of acoustic, age, and gender
information is performed in a hierarchical manner. Since the scenario of the
IEMOCAP is not explained; it is assumed the results obtained by Zhao et al.
(rows fifth and sixth) are in speaker-independent (SI) scenarios. In
\cite{Zhao2019}, the authors copied the parts of the dataset for augmentation or
balancing, since they also evaluated categorical emotion. The addition of these
data improved the CCC scores; however, this technique should be avoided since
the model learns the same data twice.

\begin{table}[htbp]
    \caption{Comparison of this study with others; SD: speaker-dependent; SI: speaker-independent; Ac: acoustic, Li: linguistic, Vi: visual}
    \begin{center}
    \begin{tabular}{p{.3cm} l l c c c c c}
    \hline
    No. & Dataset & Authors   &   Modalities    & V     &   A   &   D \\
    \hline
    \hline
    1 & IEMOCAP SD  & Atmaja &  Ac+Li & 0.596 & 0.601 & 0.499 \\
    2 & IEMOCAP SI  & Atmaja &  Ac+Li & 0.553 & 0.579 & 0.465 \\
    3 & MSPIN SD    & Atmaja &  Ac+Li & 0.486 & 0.641 & 0.524 \\
    4 & MSPIN SI    & Atmaja &  Ac+Li & 0.291 & 0.570 & 0.405 \\
    5 & IEMOCAP     & Zhao et al. \cite{Zhao2018a}  & Ac  & 0.715 & 0.392 & 0.539 \\
6 & IEMOCAP     & Zhao et al. \cite{Zhao2019}   & Ac & 0.590 & 0.689
& 0.591 \\
7 & IEMOCAP (train) \&  & AbdelWahab \&  & Ac  & 0.140	
& 0.305	& 0.181 \\
 & MSP-Podcast (test) & Busso \cite{Abdelwahab2018}\\
8 & MSP-Podcast (train) \&  & Parthasarathy \& & Ac & 0.235	& 0.623	& 0.441 \\
 & IEMOCAP (test)  & Busso \cite{Parthasarathy2019} \\
9 & MSP-Podcast SI & Sridhar et al. \cite{Sridhar2018} & Ac  & 
0.291	& 0.711    & 0.690 \\
10 & SEMAINE	& Yang \&	& Ac	& 0.506	& 0.680	& - \\
& & Hirschberg \cite{Yang2018} \\
11 & RECOLA & Bakhshi et al. \cite{Bakhshi2020} & Ac & 0.314	& 0.660	& - \\
12 & SEWA (DE)	& Schmitt et al. \cite{Schmitt2018}	& Ac & 0.489 & 0.499 & - \\
13 & SEWA (DE+HU) & Atmaja \& Akagi \cite{Atmaja2020}	& Ac+Vi	& 0.656	& 
0.680 & - \\
14 & SEWA (DE+HU)    & Chen et al. \cite{chen2017multimodal}	& Ac+Vi+Li & 
0.755 & 0.672 & - \\
    \hline
    \end{tabular}
    \end{center}
    \label{tab:compare}
\end{table}

To overcome the problem of mismatch among datasets, AbdelWahab and Busso
\cite{Abdelwahab2018} proposed domain adversarial neural network (DANN) for
acoustic emotion recognition. They obtained low CCC scores by using different
datasets for training and test, as shown in Table (row No.7). These results
were achieved using three layers of DANN. Parthasarathy
and Busso \cite{Parthasarathy2019} also took into account the problem of
generalization across datasets by proposing a semi-supervised method with
the reconstruction of intermediate feature representation that does not require
labels. One of the results, using opposite datasets for testing and test as used
by AbdelWahab and Busso, shows significant improvement on CCC scores. However,
both research \cite{Abdelwahab2018,Parthasarathy2019} showed low valence
prediction performance, which is tackled in this study.

Another way to improve valence prediction is by utilizing different
regularization for different emotion attributes, as proposed by Sridhar et al.
\cite{Sridhar2018}. However, as shown in row No. 9, the improved valence
prediction for valence is not comparable to arousal and dominance. In this
study, we achieve comparable performances among valence, arousal, and
dominance.  

Using other datasets, SEMAINE and RECOLA, comparable CCC scores were observed
between the results in these datasets and this study. Yang and Hirschberg
\cite{Yang2018} combined waveform and spectrogram for predicting valence and
arousal from speech. Although the results are shown for SEMAINE (row No. 10),
similar scores were observed for RECOLA. Using similar methods, Bakhshi
et al. \cite{Bakhshi2020} combined time and frequency information using
different networks for predicting valence and arousal. In this case, the score
of valence is about half from that of arousal.

SEWA is another dataset designed for emotion and sentiment research. Using this
dataset Schmitt et al. \cite{Schmitt2018} revealed the importance of mean and
standard deviation from GeMAPS feature set for dimensional SER. Significant
improvements were observed in the German (DE, Deutsch) sub-corpus. Atmaja and
Akagi \cite{Atmaja2020} added visual features in addition to acoustic features
to improve CCC scores. Finally, the last row in Table \ref{tab:compare} shows
that the fusion of acoustic, linguistic, and visual information attained the
highest average CCC score for dimensional SER.

The proposed methods in this study showed advantages among those other methods.
First, the bimodal acoustic-linguistic fusion doubles the amount of information
from the unimodal acoustic analysis. More data improves the effectiveness of
the SER system since the system can learn from more resources. The results
proved this hypothesis. Second, there is no need to add other modalities. Since
linguistic information could also be obtained from speech, the proposed method
only relies on speech data. Unlike audiovisual emotion recognition and addition
of age and gender information, additional modalities are needed for the fusion
method. Third, the fusion approach is performed automatically based on the data
(bottom-up approach). Although this approach has several disadvantages, the
implementation is less complicated than model-driven approach, and the results
show modest improvement from other methods. 

Several drawbacks of the evaluated methods have been found during this study. A
bottleneck between acoustic and linguistic processing is the major shortcoming
of this study. This drawback triggers a future study to predict linguistic
information from acoustic information only without the need for the
transcription.  Another challenge is to reduce the complexity of the proposed
two-stage processing. In practice, the SER system should be able to recognize
emotion within a speech in almost real-time. This requirement is difficult to
be accomplished within the current late fusion approach.

Aside from the comparison among different methods, Table \ref{tab:compare}
shows other trends in dimensional SER. First, the addition of other
non-linguistic information significantly improved the performance. This
significant improvement is the evidence for the existence of the relations
among non-linguistic information. Second, there is a mismatch among SER
datasets, which are currently being tackled by SER researchers. This problem is
a challenging opportunity for testing the proposed SER method for future
research. Finally, more modalities tend to improve SER performance. However,
per the stated objective of this study, some cases cannot provide the
measurement of other modalities. This study is intended to maximize the
performance of emotion recognition by fusing acoustic and linguistic
information. The result shows a substantial improvement for dimensional emotion
recognition from speech.


% \begin{table}[htbp]
%     \caption{Papers presented bimodal speech emotion recognition from acoustic and linguistic information fusion at INTERSPEECH 2020; WA: weighted accuracy, UA: unweighted accuracy; val: valence; aro: arousal}
%     \centering\begin{tabular}{l c c c c}
%         \hline
%         Authors &   Dataset & Emotion model & Fusion type & Results \\
% \hline 
% Chen and Zhao \cite{Chen2020}   &   IEMOCAP &  categorical & model &  WA: 71.6\%, UA: 72.6\% \\
% Liu et al. \cite{Liu2020}   &   IEMOCAP &   categorical & model     &  WA: 72.39\%, UA:
% 70.08\%\\
% Khare et al. \cite{Khare2020} & CMU-MOSEI   & categorical & model    &  WA: 66.2\%, F1:
% 78.4\% \\
% Lian et al. \cite{Lian2020} & IEMOCAP   &   categorical & feature &   WA: 82:68\% \\
% Juli\~{a}o et al. \cite{Juli2020}  & USOMS-e  & dimensional & feature &   Val: 61.0\% (UA), Aro:
% 48.8\% (UA)$^*$\\
% Yang et al. \cite{Yang2020} & USOMS-e   & dimensional &  decision & Val: 59.0\% (UA), Aro:
% 54.3\% (UA) \\
% So\u{g}ancıo\u{g}lu et al. \cite{Sogancoglu2020} & USOMS-e & dimensional &
% decision & Val: 63.7\% (UA), Aro: 57.5\% (UA)\\
% Lee et al. \cite{Lee2020a} &  IEMOCAP & categorical & model & WA: 57.9\%, UA:
% 48.7\% \\
% Shi et al. \cite{Shi2020}   & IEMOCAP &  dimensional & hierarchical  & Va: 67.12\% (UA),
% Aro: 46.13\% (UA) \\
% Khishna and Patil \cite{Krishna2020}    &   IEMOCAP & categorical & model &
% UA: 72.82\% \\
% Feng et al. \cite{Feng2020} & IEMOCAP & categorical & model & WA: 68.8\%, UA:
% 69.7\% \\
%         \hline
%     \end{tabular}
%     \label{tab:interspeech_result}
% \end{table}
