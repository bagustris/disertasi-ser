\renewcommand{\nomname}{ABSTRACT}
\markboth{\nomname}{\nomname}

\strut
\vspace{5pt}

\begin{center}
{\LARGE\bf Abstract}
\end{center}
\vspace{5pt}
\addcontentsline{toc}{chapter}{Abstract}
% from human multimodal perception to bimodal emotion recognition
Humans perceive emotion in multimodal ways. Speech is one of the sensory
modalities in which emotions can be perceived. Within speech, humans communicate
emotion through acoustic and linguistic information. In automatic emotion
recognition by computers, known as affective computing, there is a shift from
unimodal acoustic analysis to multimodal information fusion. As in human speech
emotion perception, computers should be able to perform speech emotion
recognition (SER) from bimodal acoustic-linguistic information fusion.
 
%  goal of the research
This research aims to investigate the necessity to fuse acoustic with
linguistic information for recognizing dimensional emotions. To achieve this
goal, three sub-goals were addressed: SER by using acoustic features only,
fusing acoustic and linguistic information at the feature level, and fusing
acoustic and linguistic information at the decision level.

The first strategy aims at maximizing the potency of recognizing dimensional
SER by merely using acoustic information through investigating the region of
analysis and the effect of silent pause regions. This study generalizes the
effectiveness of means and standard deviations to represent acoustic features
and the prediction of the importance of silent pause regions for dimensional
SER. In addition, the aggregation of acoustic feature models valence and
arousal prediction better than the majority voting method. Although several
approaches have been carried out, acoustic-based dimensional SER still has some
limitations. The major drawback is the low performance of valence's prediction
score.

The second and third strategies aim at improving the valence prediction,
investigating the necessity of bimodal information fusion, and evaluating the
fusion frameworks for fusing acoustic and linguistic information. Two fusion
methods for acoustic-linguistic information fusion are studied namely
early-fusion approach and late-fusion approach. At the feature level (FL) or
early-fusion approach, two fusion methods are evaluated --- feature
concatenation and network concatenation. The FL methods showed significant
performance improvements over unimodal dimensional SER.  At the decision level
(DL) or late-fusion approach, acoustic and linguistic information are trained
independently, and the results are fused by support vector machine (SVM) to
make the final predictions.  Although this proposal is more complex than the
previous FL fusion, the results showed improvements over the previous DL
approach. These studies revealed the necessity to fuse acoustic with linguistic
features for dimensional SER.

This study links the current problems in dimensional SER with its potential
solutions. The fusion of acoustic and linguistic information fills the gap in
dimensional SER. The FL approach improved the performance of unimodal SER
significantly. The DL approach improves the FL approach's performance by fusing
decisions obtained from the bimodal FL approaches. The results of this research
is expected to contribute in gaining better insights for the future strategy in
implementing SER, whether to use acoustic-only features (less complex, less
accurate), early-fusion method (more complex, more accurate), or
late-fusion method (most complex, most accurate). \\

\noindent \textbf{Keywords:} dimensional emotion, affective computing, speech
emotion recognition, information fusion, acoustic
