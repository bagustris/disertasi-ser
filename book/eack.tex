\thispagestyle{plain}
\renewcommand{\nomname}{ACKNOWLEDGMENTS}
\markboth{\nomname}{\nomname}
\phantomsection\addcontentsline{toc}{chapter}{Acknowledgments}

\strut
\vspace{20pt}

\begin{center}
{\LARGE\bf Acknowledgments}
\end{center}
\vspace{20pt}

% thank to prof akagi
The author wishes to express his sincere gratitude to his principal supervisor,
Professor Masato Akagi of Japan Advanced Institute of Science and Technology,
for his constant encouragement and kind guidance during this dissertation work.
Prof. Akagi not only guides the author to the research direction but also
pushes the timeline of the author's research to stay on track. Prof. Akagi is a
role model of supervisor for granting ``license'' for prospective researcher, a
Ph.D. candidate. 

% thank to prof Unoki
The author also wishes to express his thanks to Professor Unoki, the
co-supervisor of this dissertation. Without his critical questions, some parts
of this doctoral study will not exist. The author owes the idea of multitask
learning and silent pause feature calculations to Prof. Unoki.

% thank to prof Shirai
The author is grateful to Professor Kiyoaki Shirai for his helpful suggestions
and discussions during minor research. With the acoustic information science
background, the author has no sufficient knowledge of linguistic information
processing until conducting minor research in his lab. His patience and
kindness accelerate the author's understanding of the use of linguistic
information for dimensional speech emotion recognition.

% thank to AIS member
The kind, friendly, and warm environment at Acoustic Information Science-Lab
(Akagi and Unoki Lab) was the ideal place for conducting research and study.
Surrounded by forests and natural landscapes, there is no better place for 
Ph.D. study except in JAIST. The author would like to thank all AIS members for
their kindness support during his Ph.D. study.

% thank to MEXT
Funding is one of the crucial factors when conducting research. The author
would like to thank the Ministry of Education, Culture, Sports, Science, and
Technology (MEXT) of Japan for granting a scholarship for his studies.

Finally, the author would like to thank his family and friends. This
dissertation is dedicated to them. 

% Finally, time is short, writing is hard, and thesis is
% long. This thesis is endurance test for facing the short, the hard, and the
% long.
